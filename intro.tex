\pagenumbering{roman}
\begin{center}
	\large{\textbf{CERTIFICATE OF APPROVAL}}
\end{center}
\vspace{1cm}
This is to certify that this minor project work entitled \textbf{``Automatic Colorization of Greyscale Japanese Comics using GANs``} submitted by Abhinas Regmi(KCE077BCT005), Abi Shrestha (KCE077BCT006), Richard Khewa Limbu (KCE077BCT025), and Safal Adhikari (KCE077BCT030) has been examined and accepted in the partial fulfillment of the requirements for the degree of Bachelor in Computer Engineering. The project was carried out under special supervision and within the time frame prescribed by the
syllabus.\\
\vspace{1in}
\begin{multicols}{2}
	\begin{center}
		..........................................\\
            \textbf{Er. Kishor Kumar Adhikari, PhD}\\
            External Examiner\\
            Associate Professor  \\
            Department of Computer and Electronics Engineering\\
            National College of Engineering
                    
                  \end{center}
                
                \columnbreak
                  \begin{center}
                    ..........................................\\
                    \textbf{Er. Niranjan Bekoju}\\
                    Project Supervisor\\
                      Lecturer\\
                            Department of Computer Engineering\\
                    Khwopa College of Engineering\\
                    
                  \end{center}
                \end{multicols}
                \vspace{1in}
                \begin{center}
                  ..........................................\\
                  \textbf{Er. Dinesh Gothe}\\
                  Head of Department,\\
                  Department of Computer Engineering\\
                  Khwopa College of Engineering
                \end{center}
                \pagebreak


\large
	\chapter*{Copyright}
\normalsize
\addcontentsline{toc}{section}{Copyright}
	The author has agreed that the library, Khwopa College of Engineering  may make this report freely available for inspection. Moreover, the author has agreed that permission for the extensive copying of this project report for scholarly purpose may be granted by supervisor who supervised the project work recorded here in or, in absence the Head of The Department where in the project report was done. It is understood that the recognition will be given to the author of the report and to Department of Computer Engineering, KhCE in any use of the material of this project report. Copying or publication or other use of this report for financial gain without approval of the department and author’s written permission is prohibited. Request for the permission to copy or to make any other use of material in this report in whole or in part should be addressed to: \\
	\vspace{1cm} \\
	Head of Department \\
	Department of Computer Engineering\\
	Khwopa College of Engineering\\
	Libali,\\
	Bhaktapur, Nepal\\
\pagebreak


\large
\chapter*{Acknowledgement}
\normalsize
\addcontentsline{toc}{section}{Acknowledgement}
We take this opportunity to express our deepest and sincere gratitude to our supervisor Er. Niranjan Bekoju, for his insightful advice, motivating suggestions for this project and also for his constant encouragement and advice throughout our Bachelor’s program.\\\\
Also, we would like to thank our HoD Er. Dinesh Gothe for providing valuable suggestions and for supporting the project.
\begin{table}[h]
	\begin{tabular}{@{}ll}
		      Abhinas Regmi & KCE077BCT005\\
			Abi Shrestha & KCE077BCT006\\
			Richard Khewa Limbu & KCE077BCT025\\
			Safal Adhikari & KCE077BCT030\\
	\end{tabular}
\end{table}
\pagebreak

\large
\chapter*{Abstract}
\normalsize
\addcontentsline{toc}{section}{Abstract}
In the ever-evolving realm of entertainment, Manga and Comics, traditionally presented in the classic black and white format, are increasingly sought after in their colorized renditions. The demand for colourized Manga, underscores the importance of expediting the labour-intensive manual colorization process, prompting the exploration of innovative, automated solutions. In response to this imperative, we propose the development of an advanced model designed to streamline and accelerate the colorization of Manga and Comics.\\

\noindent Our proposed methodology utilizes unpaired datasets, to train a cycle Generative Adversarial Networks (cycleGANs) based model. Our model can learn the mapping between black and white and colourized counterparts without requiring strict image correspondences in the training set. This not only ensures the effective colorization of new Manga inputs but also tackles quintessential problems in existing methodologies. By researching the intersection of Japanese Comics, GANs, and deep learning, our proposal seeks to contribute significantly to the automated colorization landscape, fostering efficiency and creativity in the production of visually captivating Manga and Comics. \\\\

\noindent \textbf{Keywords}: \textit{Colorization, Manga, cycleGAN, Image Processing, Japanese Comics, Neural Network, Deep Learning}
\pagebreak

\tableofcontents
\addcontentsline{toc}{section}{Contents}

% \listoftables
% \addcontentsline{toc}{section}{List of Tables}
% \break
% \pagebreak

\listoffigures
\addcontentsline{toc}{section}{List of Figures}
\pagebreak

\listoftables
\addcontentsline{toc}{section}{List of Tables}
\pagebreak

\Large
\begingroup
\let\clearpage\relax
\chapter*{List of Abbreviation}
\endgroup
\normalsize
\addcontentsline{toc}{section}{List of Abbreviation}

\begin{table}[h]
	\begin{tabular}{l l}
		\textbf{Abbreviations} & \textbf{Meaning}                                               \\
		AI                     & Artificial Intilligence                                        \\
		 API     & Application Programming Interface                          \\
    CNN      & Convolutional Neural Network                              \\
    CSS      & Cascading Style Sheets                                    \\
    CUDA    & Compute Unified Device Architecture                        \\
    cGAN      & Conditional Generative Adversarial Network               \\
    GAN     & Generative Adversarial Network                             \\
    HTML     & HyperText Markup Language                                 \\
    JS    & JavaScript                                                  \\
    ML     & Machine Learning                                           \\
    NN     & Neural Network                                             \\
    RDP     & Remote Desktop Protocol                                    \\
    ResNet      & Residual Network                                        \\
    REST    & Representational State Transfer                            \\
    SSH     & Secure Socket Shell                                        \\
    BCE & Binary Cross Entropy                                         \\
    MSE & Mean Squared Error                                           \\
    PSNR & Peak Signal-to-Noise Ratio                                   \\
    SSIM & Structural Similarity Index Measure                          \\                         
	\end{tabular}
\end{table}
\pagebreak