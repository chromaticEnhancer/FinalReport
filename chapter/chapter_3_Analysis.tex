\pagenumbering{arabic}
\setcounter{page}{9}
\chapter{Analysis}

\section{Requirement Analysis}
\subsection{Hardware Requirements}
\begin{itemize}
    \item   High capacity \textbf{RAM} to handle memory-intensive tasks
    \item   \textbf{Nvidia RTX series GPU} for CUDA based computations
    \item   \textbf{SSD storage} for faster read/write speeds during image processing
    \item   Additional high-capacity \textbf{external storage} for storing large datasets and image collections
\end{itemize}

\subsection{Software Requirements}
\begin{itemize}
    \item \textbf{Python3.10} - a high-level, general-purpose programming language.
    \item \textbf{Torch} - an open source ML library used for creating deep neural networks and is written in the Lua scripting language.
    \item \textbf{Torchvision} - a library consisting of popular datasets, model architectures, and image transformations for computer vision.
    \item  \textbf{FastAPI} - a modern web framework for building RESTful APIs in Python.
    \item \textbf{HTML/CSS/Javascript} - set of languages used for creating web pages
    \item  \textbf{SSH/RDP} - network communication protocols to access the remote server for development purposes.
    \item  \textbf{VSCode} - code editor for development.
    \item  \textbf{Git/GitHub} - version control System and repository.
    \item \textbf{Ngrok} - a tool that creates a secure tunnel to the localhost
    \item \textbf{Tensorflow} - a machine learning framework for building and training neural networks.
    \item \textbf{Keras} - a high-level neural networks API running on top of TensorFlow that facilitates fast experimentation and prototyping.
    \item \textbf{Visdom} - a Python library for creating interactive visualizations and monitoring the training of machine learning model.
\end{itemize}
\pagebreak

\section{Feasibility Study}
\subsection{Economic Feasibility}
    The total expenditure of the project is just computational power. For this, the college has assisted us with a remotely accessible virtual machine that is sufficiently capable for our project.  As for the other aspects of this project, none of them pose any economic cost. The process of acquisition and preparation of dataset required to train the model is explained in the methodology section.

\subsection{Techical Feasibility}
    The dataset, acquired in accordance with the Fair Usage Policy, will be preprocessed. This involves transforming the colored images into grayscale. Given the complexity of our model and the specifications of our machine, we anticipate a training duration of approximately 60 hours.

\subsection{Operational Feasibility}
    After the model is trained, a web interface will be created for user accessibility and interaction.  Furthermore, a Python package will also be created for easy integration into other codebases.

\subsection{Legal Feasibility}
    Manga are protected by copyright as works of art, making it difficult for large dataset for training. That’s why we have gathered legally available dataset to train our model under Fair Usage Policy.
    
